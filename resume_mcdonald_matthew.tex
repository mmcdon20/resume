\documentclass[margin]{res}
\usepackage{multicol}
\usepackage[hidelinks]{hyperref}
\setlength{\leftmargini}{10pt}

\usepackage[utf8]{inputenc}
\inputencoding{latin1}
\inputencoding{utf8}

\begin{document}

  \moveleft.5\hoffset\centerline{\huge\bf Matthew R. McDonald}
  \moveleft\hoffset\vbox{\hrule width\resumewidth height 1pt}\smallskip
  \moveleft.5\hoffset\centerline{
    (773) 726-0872 | 
    \href{mailto:mmcdon20@live.com}{mmcdon20@live.com} |
    \href{http://mmcdon20.github.io/}{mmcdon20.github.io}}

\begin{resume}
                
\section{EXPERIENCE} 
  {\sl DePaul Research Assistant} (file systems forensics) \hfill June 2013 - November 2013
  \begin{itemize}\itemsep -2pt
    \item Wrote and tested file system forensics software intended to benefit law \\ enforcement agencies, which extracted and verified data from file systems so as to improve the credibility of evidence brought to court.
    \item Written in Java, Scala, Coq, and OCaml. 
  \end{itemize}
        
  {\sl Programming Competitor} (DePaul Lost Souls) \hfill September 2013 - November 2013
  \begin{itemize}\itemsep -2pt
    \item Competed in the 2013 ACM international collegiate programming contest.
    \item Collaborated with a team of 3 on solving programming challenges in Java.
    \item Placed 31st out of 129 teams. 
  \end{itemize}
                
\section{PROJECTS}
  {\sl zipper}\hfill \href{http://github.com/mmcdon20/zipper}{github.com/mmcdon20/zipper}
  \begin{itemize} \itemsep -2pt
    \item Scala library for writing collections of files to zip files, and extracting zip files.
    \item Wraps the Java standard library, and makes working with zip archives simpler.
  \end{itemize}

  {\sl Interactive Map of Crime in Chicago}\hfill \href{http://github.com/mmcdon20/chicago\_police\_map}{github.com/mmcdon20/chicago\_police\_map}
  \begin{itemize} \itemsep -2pt
    \item Ruby on Rails application that uses data sets on police beats and criminal activity from the City of Chicago data portal.
    \item Data is presented with an interactive map made with Leaflet.
  \end{itemize}
                
\section{EDUCATION} 
  {\sl Functional Programming Principles in Scala} \hfill  July 2014 \\
  Coursera / École Polytechnique Fédérale de Lausanne
  \begin{itemize} \itemsep -2pt 
    \item Course taught by Scala creator Martin Odersky.
    \item \href{https://github.com/mmcdon20/coursera/blob/master/Functional%20Programming%20Principles%202014.pdf?raw=true}{Completed with distinction.} 
  \end{itemize}

  {\sl Bachelor of Science,} Computer Science (3.95 GPA) \hfill December 2013 \\
  DePaul University - Chicago, IL 
  \begin{multicols}{2} \centerline{Relevant Courses}
    \begin{itemize} \itemsep -2pt 
      \item C++ for Programmers
      \item Computer Systems I and II
      \item Database Systems
      \item Data Structures and Algorithms in Java
      \item Design and Analysis of Algorithms
      \item Discrete Math I and II
      \item Distributed Systems
      \item Frameworks for Web Application Development
      \item Intro to Compiler Design
      \item Java for Programmers
      \item Object Oriented Software Development
      \item Object Oriented Enterprise Application Development
      \item Problem Solving for Contests
      \item Programming Language Concepts
      \item Software Projects
      \item Software Testing
      \item Technical Writing
      \item Web Development I
    \end{itemize}
  \end{multicols}

\section{SOFTWARE} 
  {\sl Programming Languages:} Java, Scala, Ruby, Python, and others \\
  {\sl Other Languages:} SQL, HTML, XML, CSS, JSON \\ 
  {\sl Tools:} Intellij Idea, RubyMine, PyCharm, Eclipse, Visual Studio, Git \\
  {\sl Libraries \& Frameworks:} Rails, Web2Py, JUnit, Bootstrap, Leaflet and more 
            
\end{resume}
\end{document}

